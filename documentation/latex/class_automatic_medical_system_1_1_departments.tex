\section{AutomaticMedicalSystem.Departments Class Reference}
\label{class_automatic_medical_system_1_1_departments}\index{AutomaticMedicalSystem::Departments@{AutomaticMedicalSystem::Departments}}
Форма для вывода информации об отделах амбулатории.  


\subsection*{Public Member Functions}
\begin{CompactItemize}
\item 
{\bf Departments} ()
\begin{CompactList}\small\item\em Конструктор формы. \item\end{CompactList}\item 
int \textbf{max\_\-id} (ref int k)\label{class_automatic_medical_system_1_1_departments_96bb73f345a34961ed32817ab1f373c5}

\item 
void {\bf Departments\_\-Load} (object sender, EventArgs e)
\begin{CompactList}\small\item\em Загрузка компонентов на форме. \item\end{CompactList}\item 
int {\bf set\_\-width\_\-column} ()
\begin{CompactList}\small\item\em Метод установления ширины столбцов. \item\end{CompactList}\item 
int {\bf set\_\-max\_\-id} (ref int k)
\begin{CompactList}\small\item\em Метод нахождения максимального значения в первой колонке. \item\end{CompactList}\item 
int \textbf{select\_\-id} ()\label{class_automatic_medical_system_1_1_departments_cc74120378f96abc4a210a10169c41b2}

\item 
void \textbf{bindingNavigatorAddNewItem\_\-Click} (object sender, EventArgs e)\label{class_automatic_medical_system_1_1_departments_6cb45d17ce7801c5b5df500fd24ed3cb}

\end{CompactItemize}


\subsection{Detailed Description}
Форма для вывода информации об отделах амбулатории. 



Definition at line 17 of file Departments.cs.

\subsection{Constructor \& Destructor Documentation}
\index{AutomaticMedicalSystem::Departments@{AutomaticMedicalSystem::Departments}!Departments@{Departments}}
\index{Departments@{Departments}!AutomaticMedicalSystem::Departments@{AutomaticMedicalSystem::Departments}}
\subsubsection[{Departments}]{\setlength{\rightskip}{0pt plus 5cm}AutomaticMedicalSystem.Departments.Departments ()}\label{class_automatic_medical_system_1_1_departments_eb1cc3edc017937b1f2e256e4cfb1f55}


Конструктор формы. 



Definition at line 27 of file Departments.cs.

\subsection{Member Function Documentation}
\index{AutomaticMedicalSystem::Departments@{AutomaticMedicalSystem::Departments}!Departments\_\-Load@{Departments\_\-Load}}
\index{Departments\_\-Load@{Departments\_\-Load}!AutomaticMedicalSystem::Departments@{AutomaticMedicalSystem::Departments}}
\subsubsection[{Departments\_\-Load}]{\setlength{\rightskip}{0pt plus 5cm}void AutomaticMedicalSystem.Departments.Departments\_\-Load (object {\em sender}, \/  EventArgs {\em e})}\label{class_automatic_medical_system_1_1_departments_3042203d4306940071a4eedf841d69e5}


Загрузка компонентов на форме. 



Definition at line 70 of file Departments.cs.\index{AutomaticMedicalSystem::Departments@{AutomaticMedicalSystem::Departments}!set\_\-max\_\-id@{set\_\-max\_\-id}}
\index{set\_\-max\_\-id@{set\_\-max\_\-id}!AutomaticMedicalSystem::Departments@{AutomaticMedicalSystem::Departments}}
\subsubsection[{set\_\-max\_\-id}]{\setlength{\rightskip}{0pt plus 5cm}int AutomaticMedicalSystem.Departments.set\_\-max\_\-id (ref int {\em k})}\label{class_automatic_medical_system_1_1_departments_2846623c15067213f1b15258be3a2d7a}


Метод нахождения максимального значения в первой колонке. 

\begin{Desc}
\item[Parameters:]
\begin{description}
\item[{\em k}]Первое сгенерированное значение\end{description}
\end{Desc}
\begin{Desc}
\item[Returns:]Измененный параметр k равный максимальному\end{Desc}


Definition at line 97 of file Departments.cs.\index{AutomaticMedicalSystem::Departments@{AutomaticMedicalSystem::Departments}!set\_\-width\_\-column@{set\_\-width\_\-column}}
\index{set\_\-width\_\-column@{set\_\-width\_\-column}!AutomaticMedicalSystem::Departments@{AutomaticMedicalSystem::Departments}}
\subsubsection[{set\_\-width\_\-column}]{\setlength{\rightskip}{0pt plus 5cm}int AutomaticMedicalSystem.Departments.set\_\-width\_\-column ()}\label{class_automatic_medical_system_1_1_departments_8e1332b338b0602dd5bef837e62c7201}


Метод установления ширины столбцов. 

\begin{Desc}
\item[Returns:]Значение ширины 3 столбца DataGridView\end{Desc}


Definition at line 82 of file Departments.cs.

The documentation for this class was generated from the following file:\begin{CompactItemize}
\item 
D:/projects/Visual Studio 2012/C\#/AutomaticMedicalSystem/AutomaticMedicalSystem/Departments.cs\end{CompactItemize}
